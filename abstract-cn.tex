\phantomsection % 让加入的目录带超链接
\addcontentsline{toc}{section}{摘要} % 在目录中显示摘要
%========================================
% 中文摘要
\renewcommand\abstractname{{\fontsize{18}{21.6} \selectfont \CJKfamily{ZS} 论文题目}}
\begin{abstract}

\begin{spacing}{2}

\centerline{{\fontsize{10.5}{10.5} \selectfont \CJKfamily{KT} xxxxxx \hspace{2pt} xxxx}}

\centerline{{\fontsize{9}{9} \selectfont \CJKfamily{KT} 学号:xxxxxx}}

\end{spacing}

\vspace{1em}

\begin{spacing}{1.3}
{\fontsize{12}{12} \selectfont \CJKfamily{KT} \textbf{【摘要】}}
{\fontsize{10.5}{10.5} \selectfont \CJKfamily{KT}
RSS的全称是Really Simple Syndication,是一种透过XML(eXtensible Markup Language)特性所制定的格式,让网站的管理者可以把网页内容传给订阅用户。这是个有点像电子报和新闻群组的东西,但是赋予读者更大的自定力。利用RSS订阅博客,新闻,社交评论后,会自动在客户端接收新消息,而不用每个站点都去查看消息了,这样子可以大大提高阅读效率,专注于阅读。本项目基于Web端,免去了安装各类型客户端的繁琐,更好地实现跨平台地使用。通过接入Pocket和印象笔记,用户能够将自己感兴趣地文章保存起来。
}

\vspace{1em}

{\fontsize{12}{12} \selectfont \CJKfamily{KT} \textbf{【关键词】}}
{\fontsize{10.5}{10.5} \selectfont \CJKfamily{KT}
RSS;HTTP;Python;MongoDB;OAuth
}
\end{spacing}

\end{abstract}

%========================================
%		Packages used in this template
\usepackage{xeCJK}			% 中文支持
\usepackage{fontspec}
\usepackage{pdfpages}		% 插入pdf
\usepackage{graphicx}		% 图形支持
\usepackage{amssymb, amsmath}	% 数学符号
\usepackage{fancyhdr}		% 页眉设置
\usepackage{lastpage}       % 引用最后一页,即算出总页数
\usepackage{cite}			% 引用文献
\usepackage{indentfirst}	% 首行缩进
\usepackage[]{hyperref}		% 让tableofcontents支持超链接
\usepackage[top=1in,bottom=1in,left=1.4in,right=1.2in]{geometry}	% 设置页边距
\usepackage{booktabs}       % 增加表格效果
\usepackage{listings} %插入代码块
\usepackage{setspace} % 调整行距
\usepackage{url}      % 参考文献引用网页
%\usepackage[top=0.8in,bottom=0.8in,left=1.2in,right=0.6in]{geometry}  %设置页边距(学校的要求)

%========================================
%		Settings
\setmainfont{Times New Roman} % 英文字体样式
\setCJKmainfont[BoldFont={SimHei}]{SimSun}      % 中文字体设置为宋体
\setCJKfamilyfont{ZenHei}{SimHei} % Use \CJKfamily{ZenHei} where you need.
\setCJKfamilyfont{ZS}{STZhongsong} % 华文中宋
\setCJKfamilyfont{KT}{KaiTi} % 楷体

\setlength{\parindent}{2em}	% 首行缩进,2字符
\numberwithin{equation}{section}   % 使公式标号为 3.1 的形式

\setlength{\parskip}{0.6\baselineskip} % 设置段间距离
\renewcommand{\baselinestretch}{1.3} % 设置行间距离

%========================================
%		Redefine commands
\renewcommand\abstractname{\Large \bfseries \CJKfamily{ZenHei}摘\ 要}		% 摘要 ,
\renewcommand{\figurename}{\CJKfamily{ZenHei} 图} 			% 图
\renewcommand{\tablename}{\CJKfamily{ZenHei} 表}            % 表
\renewcommand\refname{{\fontsize{10.5}{12.6} \selectfont \CJKfamily{KT} \textbf{【参考文献】}}} % 参考文献
\renewcommand\contentsname{\centerline{\CJKfamily{ZenHei} 目录}}	%目录居中
\renewcommand{\today}{\number\year 年 \number\month 月 \number\day 日}	%中文日期
%\renewcommand{\theequation}{\arabic{chapter}-\arabic{equation}}

%========================================
%		Header Settings
\pagestyle{fancy}			%
\chead{\textcolor{gray}{\small{深圳大学本科毕业论文——xxxxx}}}	% 页眉中部
\lhead{}		% 页眉左部,设为空
\rhead{}		% 页眉右部,设为空

%========================================

